\documentclass[10pt]{article}
\usepackage[toc]{appendix}
\begin{document}

\title{HPC Metric Generation for Node User Group}
\author{Harry Shepherd}

\maketitle
\tableofcontents

\section{Introduction}
To provided optimum used of the supercomputer for climate production runs it is important to be able to monitor the usage in a clear and concise manner. It is also important to be able to present the data in a format that is useful to the group.
A Postgres database has been recently populated with accounting details for model runs taken from the PBS scheduler.
The database can be interacted with using SQL Alchemy and Pandas via python, and so a series of scripts are presented, along with the automatic generation of a html report.


\section{Code Design}
The code for calculating the metrics and generating the website is contained within this directory. This is summarised
\begin{itemize}
 \item[] \texttt{css/} Directory containing stylesheets for the report
 \item[] \texttt{java\_script/} Directory containing java script for the report
 \item[] \texttt{make\_report.py} The top level script to be executed
 \item[] \texttt{metric*.py} A series for scripts each for a particular metric. This contains an SQL query, data processing, the generation of a segment of the html report, and the plotting of data if required. \texttt{make\_report.py} runs one or more of these scripts.
 \item[] \texttt{params.py} User editable input
 \item[] \texttt{query\_db.py} Common functions for querying the database
\end{itemize}

\section{Making a Report}
\subsection{Changing user variables}
To change the user variables, edit the \texttt{params.py} file, the variables in each of the sections are described below. \emph{Do not} edit below the `Not for user edit' line.
\subsubsection{Paramaters}
\begin{itemize}
 \item[] \texttt{DB\_URL} URL of the Postgres database
 \item[] \texttt{XCSR\_ALLOC\_NODES} Number of nodes allocated to climate on XCS
  \item[] \texttt{XCEF\_ALLOC\_NODES} Number of nodes allocated to climate on XCE and XCF
  \item[] \texttt{XCEF\_HASWELL\_NODES} The Haswell nodes on \emph{each} of XCE and XCF that are avaliable to climate science.
\end{itemize}
\subsubsection{Date Range}
\begin{itemize}
 \item[] \texttt{START\_DATE} \texttt{datetime.date} object of the start date for data query
 \item[] \texttt{END\_DATE} \texttt{datetime.date} object of the end date for data query
 \item[] \texttt{PAST\_N\_DAYS} How many days back from today do we want to collect data. For example setting this to 7 will give the past week. If this option is to be used both \texttt{START\_DATE} and \texttt{END\_DATE} must be set to \texttt{False}
\end{itemize}
\subsubsection{Reporting}
The following variables take either \texttt{True} or \texttt{False}
\begin{itemize}
 \item[] \texttt{GROUP\_E\_F} Group the data collected for XCE and XCF if set to \texttt{True}
 \item[] \texttt{MAKE\_HTML} Make HTML report
 \item[] \texttt{MAKE\_PLOTS} Make plots for those metrics that can generate them
\end{itemize}
\subsubsection{Which Metrics}
A list of boolean variables allowing a custom report to be constructed.
\subsubsection{Variables for Individual Metrics}
Some metrics require a set of variables to provide further control.
\linebreak[4]\textbf{NODE\_USE\_SUMMARY}
\begin{itemize}
 \item[] \texttt{NUG\_LIMITS\_FILE} Path to MOCI limits file for HPC allocation
 \item[] \texttt{RESOURCE\_USED\_EXE} Path to \texttt{./resource\_used} executable
\end{itemize}
\subsection{Building the report}
To make the report, simply run the command \texttt{./make\_report.py}. If the \texttt{MAKE\_HTML} option is chosen, this will create a directoroy \texttt{report} in which will be a fully self contained html report. This can be read by opening the file \texttt{report/report.html}. Any plots can be found in the directory \texttt{report/images}.
This directory can be copied as a whole to preserve the functionality.

\section{Building the documentation}
The source for this documentation can be found in \texttt{doc/README.tex}. It can be compiled using \LaTeX by running the script \texttt{doc/build\_doc.sh}, which will then overwrite the pdf file in the top level directory.

\pagebreak[4]
\appendix
\section{Data avaliable from the database}
\begin{verbatim}
uid
uuid
core_count_req
coretime_used
coretype
energy_used
exec_host
exec_vnode
exit_status
fairshare_weight
group_costcode
host_system
job_id
job_name
memory_req
node_count_req
nodetime_used
project
queue
queue_start_time
start_time
submit_time
subproject
rchar
wchar
user
walltime_used
walltime_req
\end{verbatim}



\end{document}